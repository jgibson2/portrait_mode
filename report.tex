\documentclass[10pt,twocolumn,letterpaper]{article}
\usepackage[margin=2.5cm]{geometry}
\usepackage{times,epsfig,graphicx,amsmath,amssymb}
\usepackage[breaklinks=true,colorlinks=true,bookmarks=false]{hyperref}
\date{}

%%%%%%%%%%%%%%%%

\title{CSE 559A Project Template}

\author{%
John Gibson\\
{\tt johngibson@wustl.edu}
}


\begin{document}
\maketitle

\begin{center}\textbf{Abstract}\\~\\\parbox{0.475\textwidth}{\em
    % Abstract goes here
    
    Right out roughly a one paragraph abstract that introduces your
    report and project and provides a short summary. Keep this to less
    than 15 lines.

}\end{center}

\section{Introduction}

Please follow the steps outlined below when submitting your report.

\begin{itemize}
\item Divide the paper into sections (use the grading rubric on the
  project page as a guide)
  
\item Excluding the references and acknowledgments section, your
  report should be four pages long in this two column format.

\item Use BibTeX to generate references and cite them in your text
  using ``\\cite'' commands like so~\cite{szeliski2010computer}.

\item Please number all of your sections and displayed equations---by
  using equation environments instead of equation*.  This is good
  practice as it is important for readers to be able to refer to any
  particular equation.

\item Use either single column figures like \ref{fig:onecol}, or two
  column ones like \ref{fig:twocol}. Use the [!t] option in latex to
  make sure all figures are at the top of the page. Add captions to
  all your figures.

\item Try to lay out your report so that the abstract and figure
  captions tell an ``elevator pitch'' version of your story.

\item Please use footnotes\footnote {This is what a footnote looks
    like.  It often distracts the reader from the main flow of the
    argument.} sparingly.  Indeed, try to avoid footnotes altogether
  and include necessary peripheral observations in the text (within
  parentheses, if you prefer, as in this sentence).
\end{itemize}

\begin{figure}[!t]
\begin{center}
\fbox{\rule{0pt}{3in} \rule{0.9\linewidth}{0pt}} % Place holder, replace with \includegraphics, etc.
\end{center}
   \caption{Example of caption.}
\label{fig:twocol}
\end{figure}

\begin{figure*}[!t]
\begin{center}
\fbox{\rule{0pt}{4in} \rule{0.9\linewidth}{0pt}} % Place holder, replace with \includegraphics, etc.
\end{center}
   \caption{Example of caption.}
\label{fig:onecol}
\end{figure*}


\section{Background \& Related Work}


\section{Proposed Approach}

You might want to break this up into multiple sections, or just subsections like so.

\subsection{Description of Part A}
\label{sec:parta}

\subsection{Description of Part A}
\label{sec:partb}

\section{Experimental Results}

Experimental results, or other analysis.

\section{Conclusion}

\section*{Acknowledgments}
Describe what sources you had help from, and so on. But try to put as much as possible as direct references in the body of the paper (like to say, ``adapting the code provided by the authors of~\cite{blah}).

{\small
\bibliographystyle{ieee}
\bibliography{refs} % Create file refs.bib, and run bibtex.
}

\end{document}
